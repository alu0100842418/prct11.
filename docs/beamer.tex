\documentclass{beamer}
\usepackage[utf8]{inputenc}
\usepackage{graphicx}

\newtheorem{definicion}{Definición}
\newtheorem{ejemplo}{Ejemplo}

%%%%%%%%%%%%%%%%%%%%%%%%%%%%%%%%%%%%%%%%%%%%%%%%%%%%%%%%%%%%%%%%%%%%%%%%%%%%%%%
\title[Número Pi]{Numero Pi}
\author[Sandra Beatriz Jiménez Carballos]{Sandra Beatriz Jiménez Carballos}
\date[25-04-2014]{25 de abril de 2014}
%%%%%%%%%%%%%%%%%%%%%%%%%%%%%%%%%%%%%%%%%%%%%%%%%%%%%%%%%%%%%%%%%%%%%%%%%%%%%%%

%\usetheme{Madrid}
%\usetheme{Antibes}
%\usetheme{tree}
%\usetheme{classic}

%%%%%%%%%%%%%%%%%%%%%%%%%%%%%%%%%%%%%%%%%%%%%%%%%%%%%%%%%%%%%%%%%%%%%%%%%%%%%%%
\begin{document}
  
%++++++++++++++++++++++++++++++++++++++++++++++++++++++++++++++++++++++++++++++  
\begin{frame}

  %\includegraphics[width=0.15\textwidth]{img/ullesc}
  \hspace*{7.0cm}
  %\includegraphics[width=0.16\textwidth]{img/fmatesc}
  \titlepage

  \begin{small}
    \begin{center}
     Facultad de Matemáticas \\
     Universidad de La Laguna
    \end{center}
  \end{small}

\end{frame}
%++++++++++++++++++++++++++++++++++++++++++++++++++++++++++++++++++++++++++++++  

%++++++++++++++++++++++++++++++++++++++++++++++++++++++++++++++++++++++++++++++  
\begin{frame}
  \frametitle{Índice}  
  \tableofcontents[pausesections]
\end{frame}
%++++++++++++++++++++++++++++++++++++++++++++++++++++++++++++++++++++++++++++++  


\section{Historia}
%++++++++++++++++++++++++++++++++++++++++++++++++++++++++++++++++++++++++++++++  
\begin{frame}

\frametitle{Historia}

Ya en la antigüedad, se insinuó que todos los círculos conservaban una estrecha dependencia entre el contorno y su radio pero tan sólo desde el
siglo XVII la correlación se convirtió en un dígito\cite{gibaldMLA:2009} y fue identificado con el nombre "Pi" (de periphereia, denominación que los griegos daban
al perímetro de un círculo).

\subsection{Euclides}
Euclides precisa en sus Elementos los pasos al límite necesarios e investiga un sistema consistente en doblar, al igual que Antiphon, el número de
lados de los polígonos regulares y en demostrar la convergencia del procedimiento.

\subsection{Arquímedes}
Arquímedes reúne y amplía estos resultados.\footnote{Trabajo sobre PI} Prueba que el área de un círculo es la mitad del producto de su radio por la circunferencia y que la
relación del perímetro al diámetro está comprendida entre 3,14084 y 3,14285.

\begin{definicion}
$\pi$ es:

\begin{itemize}
\item La relación entre la longitud de una circunferencia y su diámetro.
\item El área de un círculo\cite{plan} unitario.
\item El menor número real x positivo tal que $\sin(x) = 0$
\end{itemize}

\end{definicion}
\end{frame}

%++++++++++++++++++++++++++++++++++++++++++++++++++++++++++++++++++++++++++++++  

\section{Enfoque matemático}

%++++++++++++++++++++++++++++++++++++++++++++++++++++++++++++++++++++++++++++++  
\begin{frame}

\frametitle{Enfoque matemático}

A pesar de tratarse de un número irracional continúa siendo averiguada la máxima cantidad posible de decimales. Los cincuenta primeros son:
$\pi = 3,14159265358979323846264338327950288419716939937510$ 

\begin{block}{Fórmulas}
  \begin{itemize}
  \item
  Fórmula de Leibnz: \sum_{n=0}^{\infty }{{{\left(-1\right)^{n}}\over{2\,n+1}}}=\frac{1}{1} - \frac{1}{3} + \frac{1}{5} - \frac{1}{7} + \frac{1}{9} - \cdots = \frac{\pi}{4} 
  \pause

  \item
  Euler: \sum_{n=0}^{\infty }\cfrac{2^n n!^2}{(2n + 1)!}=1 + \frac{1}{3} + \frac{1 \cdot 2}{3 \cdot 5} + \frac{1 \cdot 2 \cdot 3}{3 \cdot 5 \cdot 7} + \cdots = \frac{\pi}{2} 
  \pause

  \item
  Stirling: n! \approx \sqrt{2 \pi n} \left(\frac{n}{e}\right)^n 

  \item
  Producto de Wallis: \prod_{n=1}^{\infty} \left(\frac{2n}{2n-1}\cdot\frac{2n}{2n+1}\right) = \frac{2}{1} \cdot \frac{2}{3} \cdot \frac{4}{3} \cdot \frac{4}{5} \cdot \frac{6}{5} \cdot \frac{6}{7} \cdot \frac{8}{7} \cdot \frac{8}{9} \cdots = \frac{\pi}{2} 
  \pause

  \item
  Problema de Basilea: \zeta(2) = \frac{1}{1^2} + \frac{1}{2^2} + \frac{1}{3^2} + \frac{1}{4^2} + \cdots = \frac{\pi^2}{6} 
  \pause

  \end{itemize}

\end{block}

\end{frame}
%++++++++++++++++++++++++++++++++++++++++++++++++++++++++++++++++++++++++++++++  

\section{Ejercicios}

\subsection{Una subsección}
%++++++++++++++++++++++++++++++++++++++++++++++++++++++++++++++++++++++++++++++  
\begin{frame}
\frametitle{Título de la diapositiva}

Texto de la diapositiva
\end{frame}
%++++++++++++++++++++++++++++++++++++++++++++++++++++++++++++++++++++++++++++++  

\subsection{Creación de diapositivas}

%++++++++++++++++++++++++++++++++++++++++++++++++++++++++++++++++++++++++++++++  
\begin{frame}
\frametitle{Diapositivas}

\begin{definition}
  Un ejemplo de definición
\end{definition}

\begin{example}
  \begin{itemize}
    \item <1-> Primero \pause
    \item <2-> Segundo \pause
    \item <3-> Tercero \pause
    \item <4-> Cuarto  
  \end{itemize}
\end{example}

\end{frame}
%++++++++++++++++++++++++++++++++++++++++++++++++++++++++++++++++++++++++++++++  

\subsection{Otra subseccion}
%++++++++++++++++++++++++++++++++++++++++++++++++++++++++++++++++++++++++++++++  
\begin{frame}
\frametitle{Este es otro Título}

\begin{definicion}
  Otra definición 
\end{definicion}

\begin{ejemplo}
  \begin{enumerate}
    \item
      Primero
      \pause

    \item
      Segundo 

  \end{enumerate}
\end{ejemplo}

\end{frame}
%++++++++++++++++++++++++++++++++++++++++++++++++++++++++++++++++++++++++++++++  

\section{Bibliografía}
%++++++++++++++++++++++++++++++++++++++++++++++++++++++++++++++++++++++++++++++  
\begin{frame}
  \frametitle{Bibliografía}

  \begin{thebibliography}{10}

    \beamertemplatebookbibitems
    \bibitem[Plan Estudios, 2011]{plan}  
    Documento de verificación del grado. 
    (2011) 

    \beamertemplatebookbibitems
    \bibitem[Guía Docente, 2013]{guia}  
    Guía docente. 
    (2013) 
    {\small $http://eguia.ull.es/matematicas/query.php?codigo=299341201$}

    \beamertemplatebookbibitems
    \bibitem[URL: CTAN]{latex} 
    CTAN. {\small $http://www.ctan.org/$}

  \end{thebibliography}
\end{frame}

%++++++++++++++++++++++++++++++++++++++++++++++++++++++++++++++++++++++++++++++  
\end{document}
